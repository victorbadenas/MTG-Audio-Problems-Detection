\normallinespacing

\chapter{State of the Art}

Previous work on automatic audio problems detection has been done by Pablo Alonso-Jiménez, Lluis Joglar-Ongay, Dmitry Bogdanov, 
Xavier Serra in their paper Automatic detection of audio problems for quality control in digital music distribution presented to 
the Audio Engineering Society on March 2019 in Ireland. On this paper, multiple audio defect detections were proposed as well as 
multiple audio defects were described of which some will be relevant as they will be adapted and evaluated in this paper: \\
\subsection{Phase defects}
Phase defects correspond to unfavorable relationships between both channels of audio of a stereo file. In this case, 
the relationship that is interesting for detecting it is the linear correlation between both channels as measured by the 
pearson correlation coefficient. The pearson correlation is defined to be the division of the covariance of two signals 
and the product of the standard deviation of them. That means that the range of values that the coefficient can take is [-1,1] 
where the extremes are caused by the issues that we are tracking. \\
\begin{itemize}
\item False Stereo is caused mainly by the same information being in both channels of an audio file. This can be caused by 
exporting a mono file as stereo in a Digital Audio Workstation (DAW) for instance and it is detected when the pearson coefficient 
is close to 1. \\
\item Out of Phase Stereo is caused by a sign shift between both channels of an audio file. This sign shift can be caused by 
multiple causes, the most common one in recordings is microphone placement. When summed into a single channel audio file (mono) 
some frequency components cancel due to the issue. The importance of good correlation between both channels on a stereo signal is 
seen in the amount of articles that explain the importance of phase correlation meters in audio design and audio mixing workflows. 
Some examples of those articles include [7] [4]. \\
\end{itemize}
\subsection{Silence defects}
Having too much silence at the start or the end of the file can also be considered a problem as information with no valuable 
content is being stored. For the detection of this issue the authors propose a frame by frame analysis computation of the rectified 
envelope to find the last silent frame at the start of the file and the first silent frame at the end of the file. \\
\subsection{Noise bursts defects}
Noise bursts are characterized by a succession of artifactual samples, typically originated in the codification process of an audio 
file and are part of the group that the authors denote as digital audio artifacts and that are defined by a succession of samples 
that do not correspond to a realistic sound. \\
The detection of this issue is done by thresholding the peaks of the second derivative of the of the signal. The threshold is not 
fix but rather computed using an exponential moving average (EMA) filter over the quadratic mean value of the second derivative of 
the input frame.\\
\subsection{Humming defects}
Humming is characterized by a low frequency (50-80Hz), steady, narrowband electrical noise added to the signal. This defect is very
common in analog recording and mixing gear, as it is in the neighbourhood of the frequency of the electrical supply (50-60Hz). 
The authors use an algorithm proposed by Brandt and Bitzer in their paper “Automatic detection of hum in audio signals” where they 
propose an approach based on the computation of measuring the steadiness of the frequency bins on 10-30 seconds audio segments.\\
\subsection{Click defects}
Click defects are included in the previously mentioned digital audio artifacts. They are characterized by impulsive noise that 
can come from multiple sources such as plosive sounds or codification errors. To detects this artifacts, after LPC algorithms 
are used to define the stationary part of the waveform, this is subtracted from the audio file to then compute a peak thresholding 
of the output of a matched filter of that error signal.\\
\subsection{Clipping defects}
Clipping is a phenomenon given by an audio waveform exceeding the dynamic range of the media, it being the digital or the analog 
domain. However, we are focusing on the digital domain, where clipping is perceived when surpassed the [-1, 1] range of the values. 
This can be originated from DAC not having enough dynamic range to represent the signal that is passed as an input or because of badly 
executed limiting in mastering. \\
The method proposed by the authors is detecting streams of adjacent samples that have very little variance between them and computing 
an energy threshold in order to avoid silence to be detected as clipping.\\


\newpage


